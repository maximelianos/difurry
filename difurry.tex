\documentclass[11pt,a4paper]{article}
\usepackage{alltt}
\usepackage[utf8]{inputenc}
\usepackage[english,russian]{babel}
\usepackage{amsmath}
\usepackage{amsfonts}
\usepackage{amssymb}
\usepackage{wrapfig}
\usepackage{indentfirst}
\usepackage{setspace}
\usepackage[pdftex]{graphicx}
\usepackage{caption}
\usepackage{subcaption}
\usepackage{textcomp}
\usepackage{array}
\usepackage{listings}
\usepackage{color}
\usepackage{ucs}
\usepackage{tikz} % для создания иллюстраций
\usetikzlibrary{positioning,arrows}

\usepackage{pifont}

%\usepackage{pgfplots}
%\pgfplotsset{compat=1.9}

%\topmargin = -1.5cm
%\marginparwidth = -1cm
%\marginparsep = 0pt
%\textwidth = 16cm
%\textheight = 24cm
%\oddsidemargin = 0.2cm
%\parindent = 0.5cm
%\lineskip = 0.02cm
%\parskip = 0.2cm

\usepackage{color}
\usepackage{bigints}

\usepackage{hyperref}
\hypersetup{
    colorlinks=true,
    linkcolor=blue,
    filecolor=magenta,      
    urlcolor=cyan,
}
\urlstyle{same}

% https://readthelatexmanual.wordpress.com/2016/02/21/roman-numerals-in-latex/
\newcommand{\romannum}[1]{\MakeUppercase{\romannumeral #1}}
% https://tex.stackexchange.com/questions/7032/good-way-to-make-textcircled-numbers
\newcommand*\circled[1]{\tikz[baseline=(char.base)]{\node[shape=circle,draw,inner sep=2pt] (char) {#1};}}

\newcommand{\framed}[1]{\tikz[baseline=(char.base)]{\node[shape=rectangle,draw,inner sep=4pt] (char) {#1};}}
\newcommand{\flex}{ \tikz[baseline=(char.base)]{
%\fill[black]  (-3,-5)--(-4,-6)--(-7,-3)--(-4,0)--(-4,-1)--(-6,-3)--(-5,-3)--(-4,-2)--(-4,-3)--(-5,-4)--(-4,-5)--(-3,-5);
%--(,)   (-7,-3)--(-7,-2)
\fill[black]  (-3,-6)--(-4,-6)--(-7.5,-2.5)--(-4,1)--(-4,0)--(-6,-2)--(-6,-3)--(-5,-3)--(-4,-2)--(-3,-2)--(-5,-4)--(-3,-6);
\fill[black]  (1,-1)--(2,-1)--(0,-3)--(-4,1)--(-3,1)--(0,-2)--(1,-1);
\fill[black]  (-2,3)--(0,5)--(0,4)--(-1,3)--(0,2)--(1,3)--(2,3)--(2,2)--(1,1)--(2,0)--(3,1)--(4,1)--(2,-1)--(-2,3);
%\fill[black]  (2,0)--(-1,3)--(1,5)--(2,5)--(0,3)--(1,3)--(2,4)--(3,4)--(3,3)--(2,2)--(2,1)--(4,3)--(4,2)--(2,0);
%\fill[black]  (5,6)--(2,6)--(3,7)--(5,7)--(5,9)--(6,10)--(6,7)--(9,7)--(8,6)--(6,6)--(6,4)--(5,3)--(5,6);
\fill[black]  (3,4)--(0,5)--(1,6)--(3,5)--(2,7)--(3,8)--(4,5)--(7,4)--(6,3)--(4,4)--(5,2)--(4,1)--(3,4);
} } 
\newcommand{\crossed}[1]{#1\!\!\makebox[0pt]{$\diagup$}\makebox[0pt]{$\diagdown$}\,}
\newcommand{\newgamma}{\raisebox{0.05em}{$\gamma$}}

%{\bf Alexander Berdnikov <Alexander.Berdnikov@p50.f209.n5030.z2.fidonet.org>}\\
\def\letus{%
    \mathord{\setbox0=\hbox{$\exists$}%
             \hbox{\kern 0.125\wd0%
                   \vbox to \ht0{%
                      \hrule width 0.75\wd0%
                      \vfill%
                      \hrule width 0.75\wd0}%
                   \vrule height \ht0%
                   \kern 0.125\wd0}%
           }%
    \,
}


\newcommand{\infers}{\,\to\,}
\newcommand{\textinfers}{\;\Rightarrow\;}
\newcommand{\ex}{Ex:\quad}
\newcommand{\bfend}{\textbf{--} }
%\newcommand{\smdot}{...\;}
%\newcommand{\mat}[1]{ \mkern 4mu \bar {#1} \mkern 4mu }
\newcommand{\mat}[1]{\overline{#1}}
%\newcommand{\mat}[1]{\overset{\text{\bfend}}{#1}}

\voffset = -118pt
\textheight = 820pt
\hoffset = -112pt
\textwidth = 600pt
\newcommand{\htext}{0.46\textwidth}
\newcommand{\hstext}{0.45\textwidth}

\pagestyle{empty}










\begin{document}

%\scalebox{0.8}{
\noindent
\parbox[t][][t]{0.48\textwidth}{
\setstretch{1.1}

{\centering % 1
\scalebox{1.8}{\romannum{1}.} Ур. с разделяющимися переменными
$f(y) \cdot dy = g(x) \cdot dx \infers \int f(y)\,dy = \int g(x)\,dx$\\
}\vspace{0.5em}


$\bullet$\,
\parbox[t][]{\htext}{
$y' = f(ax + by + c) \Rightarrow$ замена $y(x)=\cfrac{z(x)-ax-c}{b}$,\\
$y'(x)=\cfrac{z'(x)-a}{b}$, $z(x)=ax+by(x)+c$}

%\hrule
%\vbox to 0.5em{}
\vspace{0.5em}

{\centering  % 2
\scalebox{1.8}{\romannum{2}.} Однородные уравнения\\
$M(x,y)dx+N(x,y)dy=0$\\
\small{\it $M$ и $N$ \bfend  однор. одной и той же степени, т.е.}\\
$M(kx,ky)=k^n\cdot M(x,y)$\small{\it, то же для $N$}\\

Замена $y(x)=t(x)\cdot x$,\;\; $y'=t'x+t$,\;\; $t(x)=\cfrac{y(x)}{x}$\\
}\vspace{0.5em}

$\bullet$\;
\parbox{\htext}{$y' = f\left(\cfrac{a_1x + b_1y + c_1}{a_2x + b_2y + c_2}\right)
\raisebox{-6pt}{\vbox{\hbox{$\gets l_1$}\hbox{$\gets l_2$}}}
\;\Rightarrow\; (x_0,y_0)=l_1\cap l_2$,\\
Замена $x=u+x_0,\;\; y=v+y_0$}\vspace{1em}

$\bullet\quad y(x)=z^m(x),\; z=y^{-m}$
\vspace{0.5em}

{\centering  % 3
\scalebox{1.8}{\romannum{3}.} Линейные ур. \romannum{1} порядка\\
$y'+a(x)y=b(x)$\\
\parbox[t][]{\htext}{
\circled{\romannum{1}} Однородное $y'+a(x)y=0 \infers$\small{\it (реш.)} $y_1=y=c \cdot \ldots\;\;\forall c$\vspace{0.5em}\\
\circled{\romannum{2}} \parbox[t][]{\htext}{Метод вариации постоянной.\\ Реш. в виде $y=c(x)y_1(x)$\\}
}
$y'+a(x)y=c'y_1+cy'_1+acy_1=b \infers$\\
$c'y_1=b \infers c=\int b(x)dx/y_1(x)$\\
}\vspace{0.5em}

$\bullet$\;
\parbox[t][]{\htext}{Ур. Бернулли $y' +a(x)y = b(x)y^{\alpha} \infers$\\
$y^{-\alpha}y' + a(x)y^{1-\alpha} = b(x)$\par\smallskip
$z(x) = y^{1-\alpha}$,\; $z'(x) = y^{-\alpha}y' (1-\alpha)$ \small{\it (не забудь $y=0$)}
}\vspace{1em}

$\bullet$\;
\parbox[t][]{\htext}{Интегрируемые комбинации\\
$ydy = \cfrac{1}{2}\;dy^2$\quad\;
$\cfrac{1}{y}\;dy=d\ln|y|$\quad\;
$xdy + ydx = d(xy)$
}\vspace{1em}

$\bullet$\;
\parbox[t][]{\htext}{Ур. Рикатти $y'+a(x)y+b(x)y^2=c(x)$\\
Если $y_1(x)$ \bfend частное реш., замена $y(x)=y_1(x) + z(x)$\\
Частное решение можно искать в виде\\
$y_1 = cx^\alpha,\;\; ce^{\alpha x},\;\; ax^2+bx+c,\;\; ax+b$
}
\vspace{0.5em}

{\centering  % 4
\scalebox{1.8}{\romannum{4}.} Ур. в полных дифференциалах
$M(x,y)dx+N(x,y)dy=0$ \bfend УПД, если
$\cfrac{\partial M}{\partial y} \equiv \cfrac{\partial N}{\partial x}$\\
$\exists F:
\left\{
\begin{aligned}
& \cfrac{\partial F}{\partial x} = M(x,y) \infers F(x,y)=\int M(x,y)dx + c(y)\\
& \cfrac{\partial F}{\partial y} = N(x,y) \!\infers\! \left(\int M(x,y)dx + c(y)\right)'_y = N(x,y)
\end{aligned}
\right.
$\vspace{1em}
Отсюда найдём $c(y) = \int c'_y dy = \ldots + c_1\quad \forall c_1$\\
$dF = 0 \Leftrightarrow$ \small{\it (ответ)} $F = C \quad \forall C$\\
\small{\it (константа $c_1$ входит в $C$)}\\
}\vspace{0.5em}

$\bullet$\;
\parbox[t][]{\htext}{Интегрирующий множитель (!!! не рекомендуется !!!)\\
$\frac{\partial M}{\partial y} \neq \frac{\partial N}{\partial x}$. Ищем $m(x, y): M\,mdx+N\,mdy=0$ \bfend УПД.\\
$m=m(x),\; m(y),\; x^a y^b$
}\vspace{1em}

$\bullet$\;
\parbox[t][]{\htext}{Выделение полного дифференциала\\
$ d(xy),\; d(\frac{x}{y}),\; d(\ln|x|)$
}

}\vrule{} %*************************************************************First page second column
\parbox[t][][t]{0.48\textwidth}{

{\centering  % Не разрешенные
\circled{$\scalebox{0.06}{\flex}_{\mkern-14mu 1}$} Ур. I порядка, не разреш. отн. производной\\
$F(x,y,y')=0$\\
}\vspace{0.5em}

\circled{1}\,
\parbox[t][]{\htext}{Разрешить отн. производной. Возможен метод решения\\ относительно $x(y)$, а не $y(x)$ :)
}\vspace{1em}

\circled{2}\,
\parbox[t][]{\htext}{$F(x,y,y')$ разрешено отн. $y:\;\; y=f(x, y')$\\
Пусть $p = y' = \cfrac{dy}{dx} \infers pdx = dy = \cfrac{\partial f}{\partial x}\, dx + \cfrac{\partial f}{\partial p}\, dp \infers$\\
$M(x,p) dx + N(x,p) dp = 0$, обычное уравнение. Можно найти явное $p = p(x)$, а можно $x=x(p)$ \bfend тогда ответ\\
$\left\{
\begin{aligned}
& x=x(p)\\
& y = f(x(p),p)
\end{aligned}
\right.$
}\vspace{1em}

\circled{3}\,
\parbox[t][]{\htext}{$F(x,y,y'):\;\; x = f(y, y')$\\
Анал.: $p = y',\, \cfrac{dy}{p} = dx = d\, f(y, p) = \cfrac{\partial f}{\partial y}\, dy + \cfrac{\partial f}{\partial p}\, dp\infers\ldots$
}

\circled{4}\,
\parbox[t][]{\htext}{Нахождение особых решений
\begin{enumerate}
\item[0.] Решить исходное уравнение
\item $p$-дискриминантные кривые:
$\left\{
\begin{aligned}
& F(x,y,p)=0\\
& \partial F\, /\, \partial p = 0
\end{aligned}
\right.$\\
\small{($p = y'$)}\vspace{1em}\\
Избавляемся от $p$ в системе, ее решения \bfend\\ $y = y_1(x, c)$
\item $y_1$ является решением исх. ур.?
\item $y_1$ \bfend особое?\quad $\forall x_0\; \exists$ реш. $y_2(x):$
$\left\{
\begin{aligned}
& y_1(x_0)=y_2(x_0)\\
& y'_1(x_0)=y'_2(x_0)
\end{aligned}
\right.$\\
То есть в любой своей точке особое решение\\ должно касаться другого решения системы.

\end{enumerate}
}

\vspace{0.5em}


{\centering  % Понижение
\circled{$\scalebox{0.06}{\flex}_{\mkern-14mu 2}$} Ур. допускающие пониж. порядка производной\\
}\vspace{0.5em}

$\bullet$\,
\parbox[t][]{\htext}{
$y^{(n)}=f(x) \infers y^{(n-1)} = \int f(x)dx \infers \ldots $
}\vspace{1em}


$\bullet$\,
\parbox[t][]{\htext}{$\circled{\crossed{$y$}}: F(x, y^{(k)}, \ldots y^{(n)}) = 0 \textinfers$ замена $y^{(k)}_{(x)}=z(x)$
}\vspace{1em}

$\bullet$\,
\parbox[t][]{\htext}{$\circled{\crossed{$x$}}: F(y, y^{(1)}, \ldots y^{(n)}) = 0$\\
Замена $y' = p(y(x)),\quad y'' = p'_y \cdot y'_x = p'p$\\
\small{\it (не забудь $p'_y=\frac{dp}{dy}$\; \bfend \;$dy=0$)}
}\vspace{1em}

$\bullet$\,
\parbox[t][]{\htext}{Однородное по $y, y', \ldots y^{(n)} \textinfers$ замена $y' = yz(x)$\\
$y$ сократятся если ур. однородное
}\vspace{1em}

$\bullet$\,
\parbox[t][]{\htext}{Выделение интегрируемых комбинаций\\
$\cfrac{y^{(n)}}{y^{(n-1)}} = \left(\ln\left|\ y^{(n-1)} \right|\right)'$
}

\vspace{4em}

\small{Использовался задачник Филиппова\\
Семинарист \bfend Елена Александровна Павельева $\heartsuit$\\
\url{https://github.com/lizzardhub/difurry}\\
$\infers$ обозначает переходы между формулами\\}
}


% *************** Second page

\noindent
\parbox[t][][t]{0.48\textwidth}{
\setstretch{1.1}

{\centering % Const coeff, вариация постоянных, Эйлер
\scalebox{1.8}{\romannum{1$'$}.} Линейные однор. n-ого порядка с const coeff\\
\hspace{6em}\parbox[t][]{\htext}{
\romannum{1} \quad $y^{(n)} + a_{n-1} y^{(n-1)} \ldots a_0 y^{(0)} = 0$\\
\romannum{2} \quad $y^{(n)} + a_{n-1} y^{(n-1)} \ldots a_0 y^{(0)} = f(x)$
}
}\vspace{0.5em}

\circled{\romannum{1}}\parbox[t][]{\htext}{\noindent\:
Характеристич. уравнение: $\letus y = e^{\lambda x}$, находим $\lambda$\\
\begin{tabular}{|l|l|}
\hline
\multicolumn{1}{|c|}{корень $\lambda$} & \multicolumn{1}{c|}{функции ФСР}\\
\hline
$\mathbb{R}$ кратности $k$ & $y_1(x)=e^{\lambda x},\ldots y_k(x)=x^{k - 1}e^{\lambda x}$\\
\hline
$\mathbb{C}$ кратности $k$ & $y_1(x) = e^{ax}\cos bx \;\ldots$\\
($k$ пар $\lambda = a \pm ib$) & $y_k(x) = x^{k - 1}e^{ax}\cos bx$,\\
 & $y_{k + 1}(x) = e^{ax}\sin bx \;\ldots$\\
 & $y_{2k}(x) = x^{k - 1}e^{ax}\sin bx$\\
\hline
\end{tabular}\vspace{1em}
$y_{\text{одн}}(x) = c_1 y_1(x) + \ldots + c_n y_n(x) \quad\forall c_1, \ldots, c_n$
}

\circled{\romannum{2}}\parbox[t][]{\htext}{\noindent\, Неодн. уравнение. $y_{\text{общ}}(x) =  y_{\text{однор}} + y_{\text{частное}}(x)$\\
1. \parbox[t][]{\htext}{Первый специальный вид: $f(x) = P_k(x) e^{\newgamma x}\Rightarrow\;\\
\Rightarrow\; y_\text{ч}(x) = x^s R_k(x) e^{\newgamma x}$\\
$s = $ кратность $\newgamma$ в решениях хар. уравнения\\
$R_k$ \bfend {\it неизвестный} многочлен $k$-ой степени\vspace{0.25em}
}

2. \parbox[t][]{\htext}{Второй: $f(x) = (P_k(x)\cos \beta x + R_m(x)\sin \beta x)\cdot e^{\alpha x}\Rightarrow\;\\
\Rightarrow\; y_\text{ч}(x) = x^s (Q_n(x)\cos \beta x + T_n(x)\sin \beta x)\cdot e^{\alpha x}$\\
$n = max(k, m)$, $s = $ кратность $\newgamma = \alpha + i\beta$\\
(либо $\alpha - i\beta$, их кратности равны)\vspace{0.25em}
}

\parbox[t][]{\htext}{$f(x)$ \bfend сумма 1 и 2 спец. видов $\textinfers y_\text{ч}(x) = \sum y_{\text{спец}}$\vspace{1em}
}

\parbox[t][]{\htext}{Общий вид пр. части.\! {\bf Метод вар. постоянных}\\
{Возможно $a_0, \ldots, a_{n-1}$ \bfend функции от $x$, т.е. не const!}\\
\romannum{1} \quad $y_{\text{одн}}(x) = c_1 y_1(x) + \ldots + c_n y_n(x) \quad\forall c_1,\ldots c_n$\\
\romannum{2} \quad $\letus y(x) = c_1(x) y_1(x) + \ldots + c_n(x) y_n(x) \infers$\\
\small{\it (находим функции $c_1, c_2, \ldots c_n$ \bfend интегрируем)}\vspace{0.5em}\\
$\left\{ \begin{aligned}
& c'_1 y_1 \quad + &\ldots& + c'_n y_n = 0\\
& c'_1 y'_1 \quad + & \ldots& + c'_n y'_n = 0\\
& &\ldots&\\
& c'_1 y^{(n - 2)}_1 + & \ldots& + c'_n y^{(n - 2)}_n = 0\\
& c'_1 y^{(n - 1)}_1 + & \ldots& + c'_n y^{(n - 1)}_n = f(x)\\
\end{aligned} \right. $\\
}\vspace{1em}
}

$\bullet$\;
\parbox[t][]{\hstext}{Ур. Эйлера\\
$x^ny^{(n)} + a_{n-1}x^{n-1}y^{(n-1)} + \ldots + a_1xy' + a_0y = f(x)$
\vspace{-0.5em}
\begin{gather*}
\text{Замена:}\;\; x =
\begin{cases}
e^t, &x > 0\\
-e^t, &x < 0\\
\end{cases},\;\; t = \ln (x),\;\; y(x) = Y(t(x)) .\\
\text{Тогда } y' = Y'_t \cdot \cfrac{1}{x} \;\Rightarrow\; \underline{xy' = Y'}\\
y'' = \left(Y' \cdot \cfrac{1}{x}\right)' = Y'' \cfrac{1}{x^2} - Y' \cfrac{1}{x^2} \;\Rightarrow\; \underline{x^2 y'' = Y'' - Y'}
\end{gather*}
\vbox{\vspace{-0.5em}Получили новое уравнение $Y(t)$. Правая часть спец. вида $\Rightarrow$ рассматриваем $x>0$ и $x<0$, решаем отн. $t$.
Иначе вар. постоянных.}

}
\vspace{0.5em}

{\centering % Ф О-Лиувилля
\scalebox{1.8}{\romannum{2$'$}.} Линейные однор. ур. с переменными coeff\\
$a_n(x) y^{(n)} +\ldots a_1(x) y' + a_0(x) y = f(x)$\\
$y_1(x)$ \bfend решение, возможно, вида $e^{\alpha x}$ или многочлен\\ (его степень надо найти). Другое решение $y(x)$ найти по
формуле {\bf Остроградского-Лиувилля}\\
$\left|\begin{aligned}y_1(x)\;& y(x)\\y'_1(x)\;& y'(x)\end{aligned}\right| = ce^{-\bigintssss \cfrac{a_{n-1}(x)}{a_n(x)}\, dx} \quad \forall c$\\
}\vspace{0.5em}

}\vrule{} %********************************************Second page second column
\parbox[t][][t]{0.48\textwidth}{
{\centering % Матрицы однор
\scalebox{1.8}{\romannum{3$'$}.} Линейные системы\\
\circled{\romannum{1}}\quad $\mat{Y}' = A\mat{Y}$ -- однородные\\\vspace{0.5em}
}

$\letus \mat{Y} = \mat{\alpha} e^{\lambda t} \infers \mat{Y}' = \mat{\alpha}\lambda e^{\lambda t} \textinfers\\ (A - \lambda E)\mat{\alpha} = 0 \textinfers |A - \lambda E| = 0$\\
Находим собств. значения $\lambda$ и с. векторы $\mat{\alpha}$.\\

1) Нельзя оставлять $\mathbb{C}$ собств. значения! Делать так:\\
\small{\it (Аналогично обычным однородным линейным ур.,\\ только кратные с.з. имеют разные с.в.)}\\
$\lambda = \alpha\pm i\beta \Rightarrow \mat{Y} = \mat{\alpha} \cdot (\cos\beta t + i\sin\beta t)e^{\alpha t} = \mat{\alpha}_1 + i\mat{\alpha}_2 \Rightarrow$\\
$\mat{Y}_{\text{одн}} = c_1\mat{\alpha}_1 + c_2\mat{\alpha}_2\quad \forall c_1, c_2$\\

2) Если кол-во с. векторов меньше кратности с. значения:\\
Допустим кратность $\lambda$ равна $3$\\
 $\mat{Y} = (\mat{\alpha}t^2 + \mat{\beta}t + \mat{\newgamma}) e^{\lambda t} \infers \mat{Y}' = (\mat{\alpha} 2t + \mat{\beta})e^{\lambda t} + (\mat{\alpha}t^2 + \mat{\beta} t + \newgamma)\lambda e^{\lambda t} = (\lambda \mat{\alpha}t^2 + (2\mat{\alpha} + \lambda\mat{\beta})t + (\mat{\beta} + \lambda\mat{\newgamma}) )e^{\lambda t}   $, $A\mat{Y} = (A\mat{\alpha}t^2 + A\mat{\beta}t + A\mat{\newgamma}) e^{\lambda t}$\\
 $\letus B = A - \lambda E.\;\;\mat{Y}' = A\mat{Y} \Rightarrow \left\{ \begin{aligned}
B\mat{\alpha} &= \mat{0}\\
B\mat{\beta} &= 2\mat{\alpha}\\
B\mat{\newgamma} &= \mat{\beta}\\
\end{aligned} \right. \infers B^3\mat{\newgamma} = \mat{0} \infers\ldots$\\
Так находим все решения \bfend наборы $\{\mat{\alpha}, \mat{\beta}, \mat{\newgamma}\}$.\\


Общее решение однородного уравнения\\
$\mat{Y}_{\text{одн}} = c_1 \mat{\alpha}_1 e^{\lambda_1 t} + \ldots + c_n \mat{\alpha}_n e^{\lambda_n t}\quad \forall c_1,\ldots c_n$\\

{\centering \circled{\romannum{2}}\quad $\mat{Y}' = A\mat{Y} + \mat{F}$ -- неоднородные\\\vspace{0.5em}} % Матрицы неоднор

1.\,
\parbox[t][]{\htext}{Первый спец. вид. Если $\mat{F} = \mat{P_m}(t) \cdot e^{\newgamma t} =
\begin{pmatrix} P_{m1}(t)\\ \ldots\\ P_{mn}(t) \end{pmatrix} e^{\newgamma t}$,\\ $m = max(m_1,\ldots ,m_n) \Rightarrow$
$\mat{Y_\text{ч}} = \mat{Q}_{m+s}(t) e^{\newgamma t}$\\ ($s=$ кратность $\newgamma$)
}\vspace{0.5em}

2.\,
\parbox[t][]{\htext}{Второй спец. вид. $\mat{F} = (\mat{P_m}(t) \cos \beta t + \mat{Q_l}(t) \sin \beta t) \cdot e^{\alpha t}\\ \Rightarrow$
$\mat{Y_\text{ч}} = (\mat{R}_{k+s}(t) \cos \beta t + \mat{T}_{k+S}(t)\sin\beta t)\cdot e^{\alpha t}\\ k = max(m, l)$\\ ($s=$ кратность $\newgamma$)
}\vspace{0.5em}

\parbox[t][]{\htext}{$\mat{F}$ \bfend сумма специальных видов $\textinfers \mat{Y_\text{ч}} = \sum y_{\text{спец}}$ % Комментарии нужны?? Нет.
}\vspace{0.5em}

} %end column

\end{document}